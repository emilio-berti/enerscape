\nonstopmode{}
\documentclass[letterpaper]{book}
\usepackage[times,inconsolata,hyper]{Rd}
\usepackage{makeidx}
\usepackage[utf8]{inputenc} % @SET ENCODING@
% \usepackage{graphicx} % @USE GRAPHICX@
\makeindex{}
\begin{document}
\chapter*{}
\begin{center}
{\textbf{\huge Package `enerscape'}}
\par\bigskip{\large \today}
\end{center}
\begin{description}
\raggedright{}
\inputencoding{utf8}
\item[Type]\AsIs{Package}
\item[Title]\AsIs{What the Package Does (Title Case)}
\item[Version]\AsIs{0.1.0}
\item[Author]\AsIs{Emilio Berti}
\item[Maintainer]\AsIs{Emilio Berti }\email{emilio.berti@idiv.de}\AsIs{}
\item[Description]\AsIs{More about what it does (maybe more than one line)
Use four spaces when indenting paragraphs within the Description.}
\item[License]\AsIs{GPL-3}
\item[Encoding]\AsIs{UTF-8}
\item[LazyData]\AsIs{true}
\item[Imports]\AsIs{raster, gdalUtils, shiny}
\item[Suggests]\AsIs{terra}
\item[RoxygenNote]\AsIs{7.1.1}
\end{description}
\Rdcontents{\R{} topics documented:}
\inputencoding{utf8}
\HeaderA{calc\_work}{Calculate work from slope raster and body size}{calc.Rul.work}
%
\begin{Description}\relax
Calculate work from slope raster and body size
\end{Description}
%
\begin{Usage}
\begin{verbatim}
calc_work(
  slope = NULL,
  m = NULL,
  output_to_disk = FALSE,
  output_file = NULL,
  g = 9.80665,
  deg_to_rad = 0.01745329,
  J_to_Kcal = 4184,
  work_in_kcal = FALSE
)
\end{verbatim}
\end{Usage}
%
\begin{Arguments}
\begin{ldescription}
\item[\code{slope}] is the slope raster

\item[\code{m}] is the body mass (kg) of the species

\item[\code{output\_to\_disk}] if to write output to disk

\item[\code{output\_file}] is the output destiation, if @param output\_to\_disc == TRUE

\item[\code{g}] is the gravity force

\item[\code{deg\_to\_rad}] is the degree to radiant conversion constant

\item[\code{J\_to\_Kcal}] is the Joule to kilocalory conversion contant

\item[\code{work\_in\_kcal}] if the output should be expressed in kcal
\end{ldescription}
\end{Arguments}
%
\begin{Value}
a raster of the energetic costs of locomotion
\end{Value}
\inputencoding{utf8}
\HeaderA{dem\_to\_slope}{Calculate slope from digital elevation model}{dem.Rul.to.Rul.slope}
%
\begin{Description}\relax
Calculate slope from digital elevation model
\end{Description}
%
\begin{Usage}
\begin{verbatim}
dem_to_slope(dem, output_to_disk = FALSE, output_file = NULL)
\end{verbatim}
\end{Usage}
%
\begin{Arguments}
\begin{ldescription}
\item[\code{dem}] raster file of the digital elevation model, either a raster
or a full path location of the file

\item[\code{output\_to\_disk}] (optional) specifies if the slope raster should be also
saved to disk

\item[\code{output\_file}] (optional) specifies the location of the output file. This
must be a full path, e.g. "/home/user/Documents"
\end{ldescription}
\end{Arguments}
%
\begin{Details}\relax
If @param output\_to\_disc = FALSE, the raster output will be saved in
the temporary R folder, which is deleted at reboot. If @output\_to\_disc =
TRUE, then @output\_file must be specified.
\end{Details}
%
\begin{Value}
raster of slope in degrees
\end{Value}
\inputencoding{utf8}
\HeaderA{enerscape}{Calculate the energy landscape}{enerscape}
%
\begin{Description}\relax
Calculate the energy landscape
\end{Description}
%
\begin{Usage}
\begin{verbatim}
enerscape(dem, m, output_to_disk = FALSE, output_file = NULL, units = "J")
\end{verbatim}
\end{Usage}
%
\begin{Arguments}
\begin{ldescription}
\item[\code{dem}] raster file of the digital elevation model, either a raster
or a full path location of the file

\item[\code{m}] is the body mass (kg) of the species

\item[\code{output\_to\_disk}] (optional) specifies if the slope raster should be also
saved to disk

\item[\code{output\_file}] (optional) specifies the location of the output file. This
must be a full path, e.g. "/home/user/Documents"

\item[\code{units}] if Joules ("J") or kilocalories ("kcal)
\end{ldescription}
\end{Arguments}
%
\begin{Details}\relax
If @param output\_to\_disc = FALSE, the raster output will be saved in
the temporary R folder, which is deleted at reboot. If @output\_to\_disc =
TRUE, then @output\_file must be specified.
\end{Details}
%
\begin{Value}
raster of slope in degrees
\end{Value}
\inputencoding{utf8}
\HeaderA{hello}{Hello, World!}{hello}
%
\begin{Description}\relax
Prints 'Hello, world!'.
\end{Description}
%
\begin{Usage}
\begin{verbatim}
hello()
\end{verbatim}
\end{Usage}
%
\begin{Examples}
\begin{ExampleCode}
hello()
\end{ExampleCode}
\end{Examples}
\inputencoding{utf8}
\HeaderA{launch\_shiny}{Shiny application for enerscape}{launch.Rul.shiny}
%
\begin{Description}\relax
Shiny application for enerscape
\end{Description}
%
\begin{Usage}
\begin{verbatim}
launch_shiny()
\end{verbatim}
\end{Usage}
\inputencoding{utf8}
\HeaderA{plot\_enerscape}{Plot enerscape object}{plot.Rul.enerscape}
%
\begin{Description}\relax
Plot enerscape object
\end{Description}
%
\begin{Usage}
\begin{verbatim}
plot_enerscape(
  en,
  what = "all",
  contour = TRUE,
  n_contour = 10,
  axes = FALSE,
  max_quantile = 1
)
\end{verbatim}
\end{Usage}
%
\begin{Arguments}
\begin{ldescription}
\item[\code{en}] is the output of the enerscape() function

\item[\code{what}] specifies what to plot. Available are "work", "slope", or "all"

\item[\code{contour}] specifies if DEM contour plots should be overlayed on the plot

\item[\code{n\_contour}] specifies how many levels of DEM contours should be plotted,
if any

\item[\code{axes}] specifies if axes should be added to the plot

\item[\code{max\_quantile}] specifies the maximum quantile to be displayed, i.e.
values above this quantile are re-assigned to the maximum value within the
quantile. This is useful for plotting, as sometimes few cells have
extremely high values, due to almost vertical displacement. In such cases,
a max\_quantile = 0.99 removes these outliers.
\end{ldescription}
\end{Arguments}
\printindex{}
\end{document}
